\subsection{Limites de RoBERTa:}

1. \textbf{Dépendance aux données d'entraînement:} RoBERTa dépend fortement des données sur lesquelles il a été formé. Si les avis alimentaires contiennent des spécificités linguistiques ou des subtilités qui ne sont pas bien représentées dans les données d'entraînement, le modèle peut ne pas être aussi précis.

2. \textbf{Interprétation des résultats:} Les modèles de NLP complexes comme RoBERTa peuvent être difficiles à interpréter. Il peut être complexe de comprendre comment le modèle arrive à certaines conclusions, ce qui peut rendre délicate l'interprétation des résultats.

3. \textbf{Besoin de ressources informatiques importantes:} RoBERTa est un modèle de grande envergure, nécessitant des ressources informatiques importantes pour son utilisation. L'inférence sur de grands ensembles de données peut être gourmande en temps et en puissance de calcul.

4. \textbf{Des Textes mal Classés:}
Même si la plupart des textes sont correctement classés, il y aura toujours quelques phrases ambiguës et mal classées. Parfois, elles peuvent sembler positives, mais en réalité, elles sont négatives. De la même manière, des phrases positives peuvent parfois sembler négatives.
