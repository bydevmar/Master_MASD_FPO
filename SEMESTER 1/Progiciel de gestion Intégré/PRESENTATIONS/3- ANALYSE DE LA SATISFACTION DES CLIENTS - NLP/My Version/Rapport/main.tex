\documentclass{rapportCS}
\usepackage{lipsum}
\setlength{\headheight}{80.63582pt}
\addtolength{\topmargin}{-20.9567pt}
\title{Rapport de projet - NLP}

\begin{document}

%----------- Informations du rapport ---------

\logouniv{logos/fpo_logo.png}

\titre{L'analyse des sentiments avec NLP} % Titre du fichier

\mention{Mathématiques Appliquées pour la Science des Données} % Nom de la Mention
\trigrammemention{Master MASD} % Pour le bas de la page
\master{Master Informatique} % Nom du master
%\filiere{Filière Management de projet et Transformation} % Nom de la filière

\eleve{BOUHLALI Abdelfattah}

\dates{2023 - 2024}

% Informations tuteurs écoles
\tuteuruniv{
    \textsc{GAOU Salma} \\
    \textsc{HAMIDI Charaf} 
} 


%----------- Initialisation -------------------
        
\fairemarges %Afficher les marges
\fairepagedegarde %Créer la page de garde

%----------- Abstract -------------------
\vspace*{\stretch{1}}
\begin{center}
	\begin{abstract}
       
        Le projet fait partie d'une enquête approfondie visant à décrypter les sentiments exprimés dans les avis des consommateurs sur la plateforme Amazon, en se concentrant spécifiquement sur les produits alimentaires. Notre principal objectif était d'utiliser des techniques avancées de traitement du langage naturel (NLP) et des modèles pré-entraînés, tels que RoBERTa, pour analyser et interpréter les opinions des utilisateurs, afin d'identifier des tendances significatives et de mettre en lumière les sentiments prédominants au sein de cette catégorie de produits. 
        
        \rule{\linewidth}{0.2 mm} \\[0.4 cm]
        \begin{center}\textbf{Summary :}\end{center} 
        The project is part of a comprehensive investigation designed to decipher the sentiments expressed in consumer reviews on the Amazon platform, with a specific focus on food products. Our main objective was to utilize advanced natural language processing (NLP) techniques and pre-trained models, such as RoBERTa, to analyze and interpret user opinions, in order to identify significant trends and highlight prevailing sentiments within this product category.
    \end{abstract}
\end{center}
\vspace*{\stretch{1}}
\newpage

%------------ Table des matières ----------------

\tabledematieres % Créer la table de matières

%------------ Corps du rapport ----------------


%------------ Introduction ----------------

\section{Introduction} 

Notre projet explore les avis des consommateurs sur Amazon, en se concentrant sur les produits alimentaires. Le défi est de comprendre ce que les clients pensent vraiment. Avec tant d'avis, il est difficile de trouver les informations importantes. Nous voulons transformer ces avis en idées utiles pour aider les entreprises à améliorer leurs produits et à satisfaire les clients.


\subsection{Problème ou Question à Résoudre}
On essaie de comprendre les avis des gens sur les produits alimentaires d'Amazon. Comment les clients se sentent-ils vraiment? C'est difficile car il y a beaucoup d'avis. Notre but est de trouver des informations importantes pour aider les entreprises.

\subsection{Contexte et Motivation}
Beaucoup de gens achètent sur Amazon, et ils laissent beaucoup d'avis. Mais ces avis ne sont pas toujours faciles à comprendre. Nous voulons aider les entreprises à comprendre ce que les clients aiment et n'aiment pas.

\subsection{Objectifs du Projet et Hypothèses à Tester}
On veut utiliser des outils spéciaux pour comprendre les avis, comme RoBERTa. On espère trouver des tendances et voir ce que les clients pensent le plus. Peut-être que ces informations aideront les entreprises à faire de meilleurs produits.







\begin{itemize}
    \item Pour les initiaux : c'est votre période de stage en entreprise.
    \item Pour les alternants : c'est la période où vous êtes en charge de votre plus gros projet, avec un suivi sur le long terme. Il faudra bien prendre garde à ce que le contenu du rapport ne soit pas un copier coller du rapport intermédiaire. Notamment, la partie commune entre ces deux rapports ne doit pas dépasser 30\% du contenu.
\end{itemize}



\newpage






%------------- Commandes utiles ----------------

\section{Quelques commandes}

Voici quelques commandes utiles :



\subsection{Insertion de figures}
%------ Pour insérer et citer une image centralisée -----
% Le premier argument est le chemin pour la photo
% Le deuxième est la hauteur de la photo
% Le troisième la légende
% Le quatrième le label
Ici, je cite l'image \ref{fig:my_label} dans le texte.
\begin{figure}[h!]
    \centering
    \includegraphics[width=0.8\textwidth]{logos/react.png}
    \caption{Mettre une légende explicite à votre figure}
    \label{fig:my_label}
\end{figure}

\subsection{Insertion d'équation}
%------- Pour insérer et citer une équation --------------

\begin{equation} \label{eq: exemple}
\rho + \Delta = 42
\end{equation}

L'équation \ref{eq: exemple} est citée ici. 

\subsection{Insertion d'une référence bibliographique}
Les références (articles scientifiques, articles de journaux, blogs, pages web) doivent être mentionnées dans le texte par une balise \cite{maref} et fait le lien avec la citation incluse dans la bibliographie.


\begin{thebibliography}{15} %nb de références possibles : 15
\bibitem{maref}
	  Leslie Lamport,
	  \emph{\LaTeX: A Document Preparation System}.
	  Addison Wesley, Massachusetts,
	  2nd Edition,
	  1994.
\end{thebibliography}
\end{document}
  