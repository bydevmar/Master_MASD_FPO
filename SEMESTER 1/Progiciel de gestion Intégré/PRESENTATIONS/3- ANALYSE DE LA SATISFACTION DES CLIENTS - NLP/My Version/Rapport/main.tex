\documentclass{rapportCS}
\usepackage{lipsum}
\setlength{\headheight}{80.63582pt}
\addtolength{\topmargin}{-20.9567pt}
\title{Rapport de projet - NLP}

\begin{document}

%----------- Informations du rapport ---------

\logouniv{logos/fpo_logo.png}

\titre{L'analyse des sentiments avec NLP} % Titre du fichier

\mention{Mathématiques Appliquées pour la Science des Données} % Nom de la Mention
\trigrammemention{Master MASD} % Pour le bas de la page
\master{Master Informatique} % Nom du master
%\filiere{Filière Management de projet et Transformation} % Nom de la filière

\eleve{BOUHLALI Abdelfattah}

\dates{2023 - 2024}

% Informations tuteurs écoles
\tuteuruniv{
    \textsc{GAOU Salma} \\
    \textsc{HAMIDI Charaf} 
} 


%----------- Initialisation -------------------
        
\fairemarges %Afficher les marges
\fairepagedegarde %Créer la page de garde

%----------- Abstract -------------------
\vspace*{\stretch{1}}
\begin{center}
	\begin{abstract}
        \lipsum[1]
        \rule{\linewidth}{0.2 mm} \\[0.4 cm]
        \begin{center}\textbf{Summary :}\end{center} 
        \lipsum[2]
    \end{abstract}
\end{center}
\vspace*{\stretch{1}}
\newpage

%------------ Table des matières ----------------

\tabledematieres % Créer la table de matières

%------------ Corps du rapport ----------------


%------------ Introduction ----------------

\section{Introduction} 
% Effacer les lignes suivantes et écrire le texte souhaité
Le rapport doit faire entre 20 et 30 pages. Les figures et tableaux doivent être référencés dans le texte. 
Vous devez faire deux résumés, un en anglais, l'autre en français qui seront inclus sur la première page du document.

Nous vous rappellons que votre document présente les travaux effectués sur la période du S4 pour les M2 et fin de S2 pour les M1.
\begin{itemize}
    \item Pour les initiaux : c'est votre période de stage en entreprise.
    \item Pour les alternants : c'est la période où vous êtes en charge de votre plus gros projet, avec un suivi sur le long terme. Il faudra bien prendre garde à ce que le contenu du rapport ne soit pas un copier coller du rapport intermédiaire. Notamment, la partie commune entre ces deux rapports ne doit pas dépasser 30\% du contenu.
\end{itemize}



\newpage
\section{Section 2}

\lipsum[1-2]

\section{Section 3}
\lipsum[3-4]
\newpage





%------------- Commandes utiles ----------------

\section{Quelques commandes}

Voici quelques commandes utiles :



\subsection{Insertion de figures}
%------ Pour insérer et citer une image centralisée -----
% Le premier argument est le chemin pour la photo
% Le deuxième est la hauteur de la photo
% Le troisième la légende
% Le quatrième le label
Ici, je cite l'image \ref{fig:my_label} dans le texte.
\begin{figure}[h!]
    \centering
    \includegraphics[width=0.8\textwidth]{logos/react.png}
    \caption{Mettre une légende explicite à votre figure}
    \label{fig:my_label}
\end{figure}

\subsection{Insertion d'équation}
%------- Pour insérer et citer une équation --------------

\begin{equation} \label{eq: exemple}
\rho + \Delta = 42
\end{equation}

L'équation \ref{eq: exemple} est citée ici. 

\subsection{Insertion d'une référence bibliographique}
Les références (articles scientifiques, articles de journaux, blogs, pages web) doivent être mentionnées dans le texte par une balise \cite{maref} et fait le lien avec la citation incluse dans la bibliographie.


\begin{thebibliography}{15} %nb de références possibles : 15
\bibitem{maref}
	  Leslie Lamport,
	  \emph{\LaTeX: A Document Preparation System}.
	  Addison Wesley, Massachusetts,
	  2nd Edition,
	  1994.
\end{thebibliography}
\end{document}
  