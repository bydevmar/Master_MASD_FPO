% Introduction/introduction.tex

\section{Introduction} 

Notre projet explore les avis des consommateurs sur Amazon, en se concentrant sur les produits alimentaires. Le défi est de comprendre ce que les clients pensent vraiment. Avec tant d'avis, il est difficile de trouver les informations importantes. Nous voulons transformer ces avis en idées utiles pour aider les entreprises à améliorer leurs produits et à satisfaire les clients.


\subsection{Problème ou Question à Résoudre}
On essaie de comprendre les avis des gens sur les produits alimentaires d'Amazon. Comment les clients se sentent-ils vraiment? C'est difficile car il y a beaucoup d'avis. Notre but est de trouver des informations importantes pour aider les entreprises.

\subsection{Contexte et Motivation}
Beaucoup de gens achètent sur Amazon, et ils laissent beaucoup d'avis. Mais ces avis ne sont pas toujours faciles à comprendre. Nous voulons aider les entreprises à comprendre ce que les clients aiment et n'aiment pas.

\subsection{Objectifs du Projet et Hypothèses à Tester}

Pour notre projet, nous avons défini plusieurs objectifs clés :

\begin{enumerate}
    \item \textbf{Comprendre les Sentiments :} Utiliser des outils avancés comme RoBERTa pour analyser les sentiments exprimés dans les avis sur les produits alimentaires d'Amazon, en identifiant s'ils sont positifs, négatifs ou neutres.
    
    \item \textbf{Identifier les Tendances :} Découvrir les tendances émergentes dans les avis, y compris les préférences alimentaires, les aspects spécifiques appréciés ou critiqués, et les évolutions au fil du temps.
    
    \item \textbf{Améliorer la Pertinence :} Déterminer les aspects les plus importants pour les clients en analysant les mots clés et les expressions fréquemment utilisés dans les avis.
\end{enumerate}

\textbf{Hypothèses à Tester :}

\begin{enumerate}
    \item Nous supposons que les sentiments des clients varient en fonction des types de produits alimentaires, et nous chercherons à identifier ces variations.
    
    \item Nous pensons que certains mots-clés auront une influence significative sur la perception des produits, et nous testerons cette hypothèse en analysant leur fréquence.
    
    \item Nous anticipons que les tendances dans les avis sur les produits alimentaires évoluent avec le temps, et nous chercherons à confirmer cette hypothèse en examinant les changements au fil des mois.
\end{enumerate}