\section{Méthodologie}

\begin{frame}
    \frametitle{Étapes de la Méthodologie}
    \begin{enumerate}
        \item \textbf{Collecte de Données} :
            \begin{itemize}
                \item Collecte des avis sur les produits alimentaires d'Amazon.
                \item Utilisation de la fonction \texttt{data.info()} pour examiner la structure de l'ensemble de données.
            \end{itemize}
        \item \textbf{Informations Générales sur la Dataset} :
            \begin{itemize}
                \item Utilisation de la fonction \texttt{data.info()} pour obtenir une vue détaillée de la structure de l'ensemble de données.
                \item Analyse des attributs et types de données de chaque colonne.
            \end{itemize}
    \end{enumerate}
\end{frame}

\begin{frame}
    \frametitle{Étapes de la Méthodologie (suite)}

    \begin{enumerate}
        \setcounter{enumi}{2}
        \item \textbf{Statistiques Descriptives pour les Attributs Numériques} :
            \begin{itemize}
                \item Utilisation de la commande \texttt{data['Score'].describe()} pour obtenir des statistiques descriptives pour l'attribut "Score".
                \item Visualisation des résultats avec des graphiques.
            \end{itemize}
        \item \textbf{Vérification des Valeurs Manquantes ou d'Incohérences} :
            \begin{itemize}
                \item Utilisation de la commande \texttt{data.isnull().sum()} pour détecter les valeurs manquantes dans chaque colonne.
                \item Résumé des résultats.
            \end{itemize}
    \end{enumerate}
\end{frame}

\begin{frame}
    \frametitle{Étapes de la Méthodologie (suite)}

    \begin{enumerate}
        \setcounter{enumi}{4}
        \item \textbf{Vérifier les Valeurs Uniques dans Chaque Colonne} :
            \begin{itemize}
                \item Utilisation de la commande \texttt{data.nunique()} pour explorer la diversité des valeurs dans chaque colonne.
                \item Interprétation des résultats.
            \end{itemize}
        \item \textbf{Prétraitement des Données} :
            \begin{itemize}
                \item Élimination des lignes en double et gestion des valeurs manquantes.
            \end{itemize}
    \end{enumerate}
\end{frame}

\begin{frame}
    \frametitle{Étapes de la Méthodologie (suite)}

    \begin{enumerate}
        \setcounter{enumi}{6}
        \item \textbf{Analyse Exploratoire des Données} :
            \begin{itemize}
                \item Distribution des scores.
                \item Calcul de la moyenne et de la médiane des scores.
                \item Distribution des sentiments.
            \end{itemize}
        \item \textbf{Traitement du Texte} :
            \begin{itemize}
                \item Suppression des URL, des balises HTML, des caractères non alphabétiques.
                \item Conversion en minuscules et suppression des stopwords.
            \end{itemize}
    \end{enumerate}
\end{frame}

\begin{frame}
    \frametitle{Étapes de la Méthodologie (suite)}

    \begin{enumerate}
        \setcounter{enumi}{8}
        \item \textbf{Utilisation du Modèle Pré-entraîné RoBERTa} :
            \begin{itemize}
                \item Initialisation du modèle RoBERTa et du tokenizer.
                \item Fonction d'évaluation des scores de RoBERTa pour l'analyse de sentiments.
                \item Analyse de sentiments avec RoBERTa sur l'ensemble de données.
            \end{itemize}
        \item \textbf{Transformation et Fusion des Résultats} :
            \begin{itemize}
                \item Stockage des résultats du modèle avec les données d'origine.
                \item Exportation des résultats en CSV (\texttt{nlp\_results.csv}).
            \end{itemize}
    \end{enumerate}
\end{frame}
