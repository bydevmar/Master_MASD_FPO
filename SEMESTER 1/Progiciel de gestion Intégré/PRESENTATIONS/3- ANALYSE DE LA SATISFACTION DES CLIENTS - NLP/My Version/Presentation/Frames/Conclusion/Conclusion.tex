\section{Conclusion}

\begin{frame}
    \frametitle{Récapitulation des Principaux Résultats}
    \begin{itemize}
        \item Tendance Globale Positive : L'analyse des scores des avis suggère une tendance globale positive, avec une prédominance d'évaluations élevées, principalement de 4 et 5.
        \item Identification de Tendances Alimentaires : RoBERTa a été utilisé avec succès pour identifier des tendances émergentes dans les avis, notamment des préférences alimentaires, des aspects appréciés ou critiqués spécifiques, et des évolutions au fil du temps.
        \item Importance des Avis Positifs : Les avis positifs avec des scores élevés sont prédominants, indiquant que la satisfaction des clients est généralement élevée.
    \end{itemize}
\end{frame}

\begin{frame}
    \frametitle{Réponse à la Question Initiale}
    La question initiale visait à comprendre les avis des clients sur les produits alimentaires d'Amazon. Les résultats obtenus suggèrent que la majorité des clients expriment des sentiments positifs envers ces produits. La satisfaction semble être élevée, ce qui peut être une information précieuse pour les entreprises cherchant à améliorer leurs produits.
\end{frame}

\begin{frame}
    \frametitle{Contributions du Projet à la Connaissance du Domaine}
    \begin{itemize}
        \item Analyse Fine des Sentiments : L'utilisation de RoBERTa a permis une analyse fine des sentiments exprimés dans les avis, offrant une compréhension approfondie des opinions des clients.
        \item Identification de Tendances et de Préférences : Le projet a contribué à l'identification de tendances émergentes et de préférences alimentaires, offrant ainsi des informations utiles pour les entreprises cherchant à répondre aux attentes du marché.
        \item Utilisation de Données Réelles d'Amazon : En utilisant un ensemble de données provenant des avis réels des clients sur Amazon, le projet a contribué à une analyse basée sur des données concrètes, renforçant ainsi la validité des résultats.
    \end{itemize}
\end{frame}

\begin{frame}
    \frametitle{Conclusion Finale}
    En conclusion, ce projet a fourni des informations précieuses sur les sentiments des clients à l'égard des produits alimentaires d'Amazon, mettant en lumière des tendances, des préférences et des points forts. Ces connaissances peuvent être exploitées par les entreprises pour améliorer leurs produits et satisfaire davantage leurs clients.
\end{frame}
