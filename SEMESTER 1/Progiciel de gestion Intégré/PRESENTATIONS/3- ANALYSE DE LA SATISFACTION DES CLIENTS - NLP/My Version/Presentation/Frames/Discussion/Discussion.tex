\section{Discussion}

\begin{frame}
    \frametitle{Points forts de RoBERTa}
    \begin{itemize}
        \item Compréhension fine des sentiments : RoBERTa excelle dans la compréhension fine des sentiments exprimés dans les avis, distinguant entre positif, négatif et neutre.
        \item Capacité à identifier les tendances : RoBERTa peut découvrir les tendances émergentes dans les avis, y compris les préférences alimentaires, les aspects appréciés ou critiqués.
        \item Analyse de mots clés et d'expressions fréquemment utilisés : RoBERTa peut aider à déterminer les aspects importants en analysant les mots clés et les expressions fréquemment utilisés dans les avis.
    \end{itemize}
\end{frame}

\begin{frame}
    \frametitle{Limites de RoBERTa}
    \begin{itemize}
        \item Dépendance aux données d'entraînement : RoBERTa dépend fortement des données sur lesquelles il a été formé, ce qui peut affecter sa précision.
        \item Interprétation des résultats : Les modèles NLP complexes comme RoBERTa peuvent être difficiles à interpréter, rendant délicate l'interprétation des résultats.
        \item Besoin de ressources informatiques importantes : RoBERTa nécessite des ressources informatiques importantes pour son utilisation.
        \item Des Textes mal Classés : Malgré la plupart des classifications correctes, il peut y avoir des textes mal classés en raison de nuances linguistiques.
    \end{itemize}
\end{frame}

\begin{frame}
    \frametitle{Suggestions et améliorations possibles du projet}
    \begin{itemize}
        \item Exploration de sous-catégories alimentaires : Explorer les sentiments dans des sous-catégories spécifiques de produits alimentaires pour obtenir des informations plus détaillées.
        \item Enrichissement du modèle avec des données spécifiques au domaine : Utiliser des données spécifiques au domaine alimentaire pour enrichir le modèle.
        \item Intégration de la rétroaction des entreprises : Permettre aux entreprises de répondre aux avis pourrait offrir une perspective plus complète sur la satisfaction du client.
        \item Analyse de la dynamique temporelle : Investiguer les tendances au fil du temps en analysant les changements mensuels dans les avis.
        \item Comparaison avec d'autres modèles NLP : Comparer avec d'autres modèles NLP pour évaluer la performance relative.
    \end{itemize}
\end{frame}

\begin{frame}
    \frametitle{Implications et conclusions tirées des résultats obtenus}
    \begin{itemize}
        \item Tendance positive globale : La distribution des scores suggère une tendance positive globale dans les avis des clients sur les produits alimentaires d'Amazon.
        \item Importance des avis avec des scores élevés : La majorité des scores sont entre 4 et 5, soulignant l'importance des avis positifs.
        \item Prévalence des évaluations positives : La fréquence élevée des évaluations positives, en particulier avec un score de 5, peut indiquer une propension des utilisateurs à partager leurs expériences positives.
    \end{itemize}
\end{frame}
